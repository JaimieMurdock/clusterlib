\documentclass{beamer}
\usetheme{CambridgeUS}

\title{Categorization}
\author{Jaimie Murdock}
\institute[IU COGS]{
    IU Cognitive Science Program\\
    810 Eigenmann Hall\\
    \texttt{jammurdo@indiana.edu}
    }
\date{November 15, 2011}

\begin{document}
% Title Page
\frame{\titlepage}

\section{Introduction}
\begin{frame}
"Issues related to concepts and categorization are nearly ubiquitous in
psychology because of people's natural tendancy to perceive a thing as
something." - Goldstone \& Kersten 2003
\end{frame}

\begin{frame}
\frametitle{Different names for the same thing...}
\begin{itemize}
  \item{Categorization}
  \item{Classification \tiny{(machine learning)}}
  \item{Clustering \tiny{(data mining)}}
  \item{Partitioning \tiny{(mathematics)}}
  \pause
  \item{Chunking \tiny{(memory)}}
  \item{Object Recognition \tiny{(vision)}}
  \item{Semantics \tiny{(linguistics)}}
  \item{Named Entity Recognition \tiny{(natural language processing)}}
\end{itemize}

\end{frame}

\subsection{Definitions}
\begin{frame}
\frametitle{What is categorization?}

\begin{itemize}
  \item{The assignment of concepts to categories}
  \item{``Seeing something as X'' - Wittgenstein, \textit{Philosophical Investigations}}
  
  \pause
  \bigskip

  \item{\textbf{What is a concept?}\\
        Whatever psychological state signifies thoughts of X}
  \item{\textbf{What is a category?}\\
        All entities that are appropriately categorized as X}
\end{itemize}
\end{frame}

\subsection{Models}
\begin{frame}
\frametitle{Prototypes vs. Exemplars}
\begin{columns}
\begin{column}{.5\textwidth}
\textbf{Prototype Model}\\
Do concepts determine categories? (Lakoff 1987)
\end{column}
\pause
\begin{column}{.5\textwidth}
\textbf{Exemplar Model}\\
Do categories determine concepts? (Nosofsky 1984)
\end{column}
\end{columns}
\end{frame}

\subsection{Utility}
\begin{frame}
\frametitle{Why do we categorize?}
\begin{itemize}
  \item{Components of thought}
  \pause
  \item{Inductive Predictions}
  \pause
  \item{Communication}
  \pause
  \item{Cognitive Economy}
\end{itemize}
\end{frame}

\begin{frame}
\frametitle{Equivilence Classes}
Distinguishable stimuli can become treated as the same thing once they are placed in the same category (Sidman 1994)\\

\bigskip
\pause

\textbf{Biology} - taxonomy (kingdom, phylum, class, order, family, genus, species)\\

\bigskip
\pause

\textbf{Things to remove from a burning house} - photographs, children, pets\\

\bigskip
\pause

Equivilence classes may not be uniquely human - sea lions (Schusterman, Reichmuth, Kastak 2000)
\end{frame}

\subsection{Representations}
\begin{frame}
\frametitle{Representations}
How are categories represented?

\begin{itemize}
  \item{rules}
  \pause
  \item{exemplars}
  \pause
  \item{prototypes}
  \pause
  \item{boundaries}
% TODO: Insert Figure 22.1
\end{itemize}
\end{frame}

\section{Algorithms}
\subsection{Introduction}
\begin{frame}
\frametitle{Algorithms}
A process to assign concepts to categories\\

\pause

\begin{itemize}
  \item{k-means Nearest Neighbors \tiny{(MacQueen 1967)}}
  \item{QT-clust \tiny{(Heyer et al. 1999)}}
  \item{Information Theoretic Clustering \tiny{(Gokcay \& Princip\'{e} 2002)}}
\end{itemize}
\end{frame}

\begin{frame}
\frametitle{Components}
Two key decisions in clustering:
\begin{columns}
\begin{column}{.5\textwidth}
\textbf{distance function}
\begin{itemize}
  \item Euclidean distance
  \item semantic similarity
  \item cross-entropy
\end{itemize}
\end{column}
\begin{column}{.5\textwidth}
\textbf{cluster assignment}
\begin{itemize}
  \item nearest neighbor
  \item minimal diameter
  \item maximize cross-cluster distance
\end{itemize}
\end{column}
\end{columns}
\end{frame}

\subsection{k-means Nearest Neighbors}
\begin{frame}
\frametitle{k-means Nearest Neighbors}
Given $n$ items, place into $k$ groups\\

\bigskip

\textbf{Initialize:} Pick $k$ centroids\\
\textbf{Assign:} Assign items to nearest centroid\\
\textbf{Update:} Recalculate centroids\\

\bigskip

Repeat until convergence of assignment

\end{frame}

\begin{frame}
% TODO: Insert images of clustering process
\end{frame}

% TODO: Insert code walkthrough

\subsection{QT-clust}
\begin{frame}
\frametitle{QT-clust}
Given $n$ items, place into groups of $\epsilon$ diameter\\

\bigskip

\textbf{Build:} for each $i \in n$, build candidate cluster $C_i$ \\
\textbf{Select:} pick largest $C_i$, remove elements from popuation \\

\bigskip

Repeat until all items are assigned.

\end{frame}

% TODO: Insert images of clustering process
% TODO: Insert code walkthrough

\subsection{Information Theoretic}
\begin{frame}
Given $n$ items, place into $k$ groups, minimizing the value of the cross-entropy function (CEF) \\

\bigskip

\textbf{Initialize:} Assign all items to random clusters
\textbf{Group:} for each $i \in n$, build group $G_i$ of size $M = n / k$ \\
\textbf{Reassign:} for each $i \in n$, see if switching $G_i$ reduces $CEF$, permanently switch cluster assignment for $x \in G_i$ which minimizes $CEF$ \\

\bigskip

Repeat until $CEF$ reaches minima
\end{frame}

% TODO: Insert images of clustering process
% TODO: Insert code walkthrough

\end{document}
